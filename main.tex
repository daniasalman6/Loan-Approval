% Options for packages loaded elsewhere
\PassOptionsToPackage{unicode}{hyperref}
\PassOptionsToPackage{hyphens}{url}
%
\documentclass[
]{article}
\usepackage{amsmath,amssymb}
\usepackage{iftex}
\ifPDFTeX
  \usepackage[T1]{fontenc}
  \usepackage[utf8]{inputenc}
  \usepackage{textcomp} % provide euro and other symbols
\else % if luatex or xetex
  \usepackage{unicode-math} % this also loads fontspec
  \defaultfontfeatures{Scale=MatchLowercase}
  \defaultfontfeatures[\rmfamily]{Ligatures=TeX,Scale=1}
\fi
\usepackage{lmodern}
\ifPDFTeX\else
  % xetex/luatex font selection
\fi
% Use upquote if available, for straight quotes in verbatim environments
\IfFileExists{upquote.sty}{\usepackage{upquote}}{}
\IfFileExists{microtype.sty}{% use microtype if available
  \usepackage[]{microtype}
  \UseMicrotypeSet[protrusion]{basicmath} % disable protrusion for tt fonts
}{}
\makeatletter
\@ifundefined{KOMAClassName}{% if non-KOMA class
  \IfFileExists{parskip.sty}{%
    \usepackage{parskip}
  }{% else
    \setlength{\parindent}{0pt}
    \setlength{\parskip}{6pt plus 2pt minus 1pt}}
}{% if KOMA class
  \KOMAoptions{parskip=half}}
\makeatother
\usepackage{xcolor}
\usepackage[margin=1in]{geometry}
\usepackage{color}
\usepackage{fancyvrb}
\newcommand{\VerbBar}{|}
\newcommand{\VERB}{\Verb[commandchars=\\\{\}]}
\DefineVerbatimEnvironment{Highlighting}{Verbatim}{commandchars=\\\{\}}
% Add ',fontsize=\small' for more characters per line
\usepackage{framed}
\definecolor{shadecolor}{RGB}{248,248,248}
\newenvironment{Shaded}{\begin{snugshade}}{\end{snugshade}}
\newcommand{\AlertTok}[1]{\textcolor[rgb]{0.94,0.16,0.16}{#1}}
\newcommand{\AnnotationTok}[1]{\textcolor[rgb]{0.56,0.35,0.01}{\textbf{\textit{#1}}}}
\newcommand{\AttributeTok}[1]{\textcolor[rgb]{0.13,0.29,0.53}{#1}}
\newcommand{\BaseNTok}[1]{\textcolor[rgb]{0.00,0.00,0.81}{#1}}
\newcommand{\BuiltInTok}[1]{#1}
\newcommand{\CharTok}[1]{\textcolor[rgb]{0.31,0.60,0.02}{#1}}
\newcommand{\CommentTok}[1]{\textcolor[rgb]{0.56,0.35,0.01}{\textit{#1}}}
\newcommand{\CommentVarTok}[1]{\textcolor[rgb]{0.56,0.35,0.01}{\textbf{\textit{#1}}}}
\newcommand{\ConstantTok}[1]{\textcolor[rgb]{0.56,0.35,0.01}{#1}}
\newcommand{\ControlFlowTok}[1]{\textcolor[rgb]{0.13,0.29,0.53}{\textbf{#1}}}
\newcommand{\DataTypeTok}[1]{\textcolor[rgb]{0.13,0.29,0.53}{#1}}
\newcommand{\DecValTok}[1]{\textcolor[rgb]{0.00,0.00,0.81}{#1}}
\newcommand{\DocumentationTok}[1]{\textcolor[rgb]{0.56,0.35,0.01}{\textbf{\textit{#1}}}}
\newcommand{\ErrorTok}[1]{\textcolor[rgb]{0.64,0.00,0.00}{\textbf{#1}}}
\newcommand{\ExtensionTok}[1]{#1}
\newcommand{\FloatTok}[1]{\textcolor[rgb]{0.00,0.00,0.81}{#1}}
\newcommand{\FunctionTok}[1]{\textcolor[rgb]{0.13,0.29,0.53}{\textbf{#1}}}
\newcommand{\ImportTok}[1]{#1}
\newcommand{\InformationTok}[1]{\textcolor[rgb]{0.56,0.35,0.01}{\textbf{\textit{#1}}}}
\newcommand{\KeywordTok}[1]{\textcolor[rgb]{0.13,0.29,0.53}{\textbf{#1}}}
\newcommand{\NormalTok}[1]{#1}
\newcommand{\OperatorTok}[1]{\textcolor[rgb]{0.81,0.36,0.00}{\textbf{#1}}}
\newcommand{\OtherTok}[1]{\textcolor[rgb]{0.56,0.35,0.01}{#1}}
\newcommand{\PreprocessorTok}[1]{\textcolor[rgb]{0.56,0.35,0.01}{\textit{#1}}}
\newcommand{\RegionMarkerTok}[1]{#1}
\newcommand{\SpecialCharTok}[1]{\textcolor[rgb]{0.81,0.36,0.00}{\textbf{#1}}}
\newcommand{\SpecialStringTok}[1]{\textcolor[rgb]{0.31,0.60,0.02}{#1}}
\newcommand{\StringTok}[1]{\textcolor[rgb]{0.31,0.60,0.02}{#1}}
\newcommand{\VariableTok}[1]{\textcolor[rgb]{0.00,0.00,0.00}{#1}}
\newcommand{\VerbatimStringTok}[1]{\textcolor[rgb]{0.31,0.60,0.02}{#1}}
\newcommand{\WarningTok}[1]{\textcolor[rgb]{0.56,0.35,0.01}{\textbf{\textit{#1}}}}
\usepackage{graphicx}
\makeatletter
\newsavebox\pandoc@box
\newcommand*\pandocbounded[1]{% scales image to fit in text height/width
  \sbox\pandoc@box{#1}%
  \Gscale@div\@tempa{\textheight}{\dimexpr\ht\pandoc@box+\dp\pandoc@box\relax}%
  \Gscale@div\@tempb{\linewidth}{\wd\pandoc@box}%
  \ifdim\@tempb\p@<\@tempa\p@\let\@tempa\@tempb\fi% select the smaller of both
  \ifdim\@tempa\p@<\p@\scalebox{\@tempa}{\usebox\pandoc@box}%
  \else\usebox{\pandoc@box}%
  \fi%
}
% Set default figure placement to htbp
\def\fps@figure{htbp}
\makeatother
\setlength{\emergencystretch}{3em} % prevent overfull lines
\providecommand{\tightlist}{%
  \setlength{\itemsep}{0pt}\setlength{\parskip}{0pt}}
\setcounter{secnumdepth}{-\maxdimen} % remove section numbering
\usepackage{booktabs}
\usepackage{longtable}
\usepackage{array}
\usepackage{multirow}
\usepackage{wrapfig}
\usepackage{float}
\usepackage{colortbl}
\usepackage{pdflscape}
\usepackage{tabu}
\usepackage{threeparttable}
\usepackage{threeparttablex}
\usepackage[normalem]{ulem}
\usepackage{makecell}
\usepackage{xcolor}
\usepackage{bookmark}
\IfFileExists{xurl.sty}{\usepackage{xurl}}{} % add URL line breaks if available
\urlstyle{same}
\hypersetup{
  pdftitle={Data Science Project - Loan Approval},
  pdfauthor={Alex Biuckians, Armin Soroosh, Dania Salman, Kaipu Liu},
  hidelinks,
  pdfcreator={LaTeX via pandoc}}

\title{Data Science Project - Loan Approval}
\author{Alex Biuckians, Armin Soroosh, Dania Salman, Kaipu Liu}
\date{}

\begin{document}
\maketitle

{
\setcounter{tocdepth}{3}
\tableofcontents
}
\subsection{Loading the Dataset}\label{loading-the-dataset}

\begin{Shaded}
\begin{Highlighting}[]
\CommentTok{\# reading dataset}
\NormalTok{loan\_df }\OtherTok{\textless{}{-}} \FunctionTok{data.frame}\NormalTok{(}\FunctionTok{read.csv}\NormalTok{(}\StringTok{"loan\_approval\_dataset.csv"}\NormalTok{))}
\end{Highlighting}
\end{Shaded}

\begin{Shaded}
\begin{Highlighting}[]
\CommentTok{\# exploring dataset}
\FunctionTok{str}\NormalTok{(loan\_df)}
\end{Highlighting}
\end{Shaded}

\begin{verbatim}
## 'data.frame':    4269 obs. of  13 variables:
##  $ loan_id                 : int  1 2 3 4 5 6 7 8 9 10 ...
##  $ no_of_dependents        : int  2 0 3 3 5 0 5 2 0 5 ...
##  $ education               : chr  " Graduate" " Not Graduate" " Graduate" " Graduate" ...
##  $ self_employed           : chr  " No" " Yes" " No" " No" ...
##  $ income_annum            : int  9600000 4100000 9100000 8200000 9800000 4800000 8700000 5700000 800000 1100000 ...
##  $ loan_amount             : int  29900000 12200000 29700000 30700000 24200000 13500000 33000000 15000000 2200000 4300000 ...
##  $ loan_term               : int  12 8 20 8 20 10 4 20 20 10 ...
##  $ cibil_score             : int  778 417 506 467 382 319 678 382 782 388 ...
##  $ residential_assets_value: int  2400000 2700000 7100000 18200000 12400000 6800000 22500000 13200000 1300000 3200000 ...
##  $ commercial_assets_value : int  17600000 2200000 4500000 3300000 8200000 8300000 14800000 5700000 800000 1400000 ...
##  $ luxury_assets_value     : int  22700000 8800000 33300000 23300000 29400000 13700000 29200000 11800000 2800000 3300000 ...
##  $ bank_asset_value        : int  8000000 3300000 12800000 7900000 5000000 5100000 4300000 6000000 600000 1600000 ...
##  $ loan_status             : chr  " Approved" " Rejected" " Rejected" " Rejected" ...
\end{verbatim}

The dataset has \textbf{4269 observations} with \textbf{13 variables}

\subsection{Data Cleaning \& Summary}\label{data-cleaning-summary}

\subsubsection{subsetting data and converting categorical variables to
factor}\label{subsetting-data-and-converting-categorical-variables-to-factor}

Removing loan\_id column since it only serves as a unique identifier and
does not contribute to the analysis. Furthermore, converting several
numeric variables into factors, as they represent categorical
information rather than continuous numerical values.

\begin{Shaded}
\begin{Highlighting}[]
\NormalTok{loan\_df }\OtherTok{\textless{}{-}} \FunctionTok{subset}\NormalTok{(loan\_df, }\AttributeTok{select =} \SpecialCharTok{{-}}\FunctionTok{c}\NormalTok{(loan\_id))}

\CommentTok{\# converting the numeric variables to factor variables}
\NormalTok{loan\_df}\SpecialCharTok{$}\NormalTok{no\_of\_dependents }\OtherTok{=} \FunctionTok{as.factor}\NormalTok{(loan\_df}\SpecialCharTok{$}\NormalTok{no\_of\_dependents)}
\NormalTok{loan\_df}\SpecialCharTok{$}\NormalTok{education }\OtherTok{=} \FunctionTok{as.factor}\NormalTok{(loan\_df}\SpecialCharTok{$}\NormalTok{education)}
\NormalTok{loan\_df}\SpecialCharTok{$}\NormalTok{self\_employed }\OtherTok{=} \FunctionTok{as.factor}\NormalTok{(loan\_df}\SpecialCharTok{$}\NormalTok{self\_employed)}
\NormalTok{loan\_df}\SpecialCharTok{$}\NormalTok{loan\_status }\OtherTok{=} \FunctionTok{as.factor}\NormalTok{(loan\_df}\SpecialCharTok{$}\NormalTok{loan\_status)}
\FunctionTok{str}\NormalTok{(loan\_df)}
\end{Highlighting}
\end{Shaded}

\begin{verbatim}
## 'data.frame':    4269 obs. of  12 variables:
##  $ no_of_dependents        : Factor w/ 6 levels "0","1","2","3",..: 3 1 4 4 6 1 6 3 1 6 ...
##  $ education               : Factor w/ 2 levels " Graduate"," Not Graduate": 1 2 1 1 2 1 1 1 1 2 ...
##  $ self_employed           : Factor w/ 2 levels " No"," Yes": 1 2 1 1 2 2 1 2 2 1 ...
##  $ income_annum            : int  9600000 4100000 9100000 8200000 9800000 4800000 8700000 5700000 800000 1100000 ...
##  $ loan_amount             : int  29900000 12200000 29700000 30700000 24200000 13500000 33000000 15000000 2200000 4300000 ...
##  $ loan_term               : int  12 8 20 8 20 10 4 20 20 10 ...
##  $ cibil_score             : int  778 417 506 467 382 319 678 382 782 388 ...
##  $ residential_assets_value: int  2400000 2700000 7100000 18200000 12400000 6800000 22500000 13200000 1300000 3200000 ...
##  $ commercial_assets_value : int  17600000 2200000 4500000 3300000 8200000 8300000 14800000 5700000 800000 1400000 ...
##  $ luxury_assets_value     : int  22700000 8800000 33300000 23300000 29400000 13700000 29200000 11800000 2800000 3300000 ...
##  $ bank_asset_value        : int  8000000 3300000 12800000 7900000 5000000 5100000 4300000 6000000 600000 1600000 ...
##  $ loan_status             : Factor w/ 2 levels " Approved"," Rejected": 1 2 2 2 2 2 1 2 1 2 ...
\end{verbatim}

After conversion, the dataset contains four categorical variables:

\begin{itemize}
\item
  \textbf{no\_of\_dependents:} Number of dependents the applicant has
  (\texttt{6} categories: \texttt{0}, \texttt{1}, \texttt{2},
  \texttt{3}, \texttt{4}, \texttt{5})
\item
  \textbf{education:} Applicant's level of education (\texttt{2}
  categories: \texttt{Graduate}, \texttt{Not\ Graduate})
\item
  \textbf{self\_employed:} Whether the applicant is self-employed
  (\texttt{2} categories: \texttt{Yes}, \texttt{No})
\item
  \textbf{loan\_status:} Loan approval status --- this is the target
  variable (\texttt{2} categories: \texttt{Approved}, \texttt{Rejected})
\end{itemize}

\subsubsection{Finding NA values}\label{finding-na-values}

\begin{Shaded}
\begin{Highlighting}[]
\FunctionTok{sum}\NormalTok{(}\FunctionTok{is.na}\NormalTok{(loan\_df))}
\end{Highlighting}
\end{Shaded}

\begin{verbatim}
## [1] 0
\end{verbatim}

There are no NA values in the data set

\subsubsection{Summary of the dataset}\label{summary-of-the-dataset}

\begin{Shaded}
\begin{Highlighting}[]
\FunctionTok{summary}\NormalTok{(loan\_df)}
\end{Highlighting}
\end{Shaded}

\begin{verbatim}
##  no_of_dependents         education    self_employed  income_annum    
##  0:712             Graduate    :2144    No :2119     Min.   : 200000  
##  1:697             Not Graduate:2125    Yes:2150     1st Qu.:2700000  
##  2:708                                               Median :5100000  
##  3:727                                               Mean   :5059124  
##  4:752                                               3rd Qu.:7500000  
##  5:673                                               Max.   :9900000  
##   loan_amount         loan_term     cibil_score  residential_assets_value
##  Min.   :  300000   Min.   : 2.0   Min.   :300   Min.   : -100000        
##  1st Qu.: 7700000   1st Qu.: 6.0   1st Qu.:453   1st Qu.: 2200000        
##  Median :14500000   Median :10.0   Median :600   Median : 5600000        
##  Mean   :15133450   Mean   :10.9   Mean   :600   Mean   : 7472617        
##  3rd Qu.:21500000   3rd Qu.:16.0   3rd Qu.:748   3rd Qu.:11300000        
##  Max.   :39500000   Max.   :20.0   Max.   :900   Max.   :29100000        
##  commercial_assets_value luxury_assets_value bank_asset_value  
##  Min.   :       0        Min.   :  300000    Min.   :       0  
##  1st Qu.: 1300000        1st Qu.: 7500000    1st Qu.: 2300000  
##  Median : 3700000        Median :14600000    Median : 4600000  
##  Mean   : 4973155        Mean   :15126306    Mean   : 4976692  
##  3rd Qu.: 7600000        3rd Qu.:21700000    3rd Qu.: 7100000  
##  Max.   :19400000        Max.   :39200000    Max.   :14700000  
##     loan_status  
##   Approved:2656  
##   Rejected:1613  
##                  
##                  
##                  
## 
\end{verbatim}

Overall, the dataset appears well-structured and balanced.
\textbf{Categorical variables} such as \texttt{education},
\texttt{self\_employment}, and \texttt{loan\_status} show nearly
\textbf{even distributions} across their categories. Most
\textbf{numerical variables} fall within reasonable ranges, with average
loan terms around \texttt{11\ years} and CIBIL scores centered near
\texttt{600}, both indicating realistic applicant profiles. However,
some financial variables like income, loan amount, and asset values show
wide variation and possible outliers. In particular, the presence of
negative values in \texttt{residential\_assets\_value} points to
potential data quality issues that will need correction before
exploration or testing.

\subsubsection{Cleaning up negative
values}\label{cleaning-up-negative-values}

Checking the number of rows that have negative values for
\texttt{residential\_assets\_value} column

\begin{Shaded}
\begin{Highlighting}[]
\FunctionTok{sum}\NormalTok{(loan\_df}\SpecialCharTok{$}\NormalTok{residential\_assets\_value }\SpecialCharTok{\textless{}} \DecValTok{0}\NormalTok{)}
\end{Highlighting}
\end{Shaded}

\begin{verbatim}
## [1] 28
\end{verbatim}

There are \texttt{28} entries with negative values.

\subsubsection{Removing entries with negative
values}\label{removing-entries-with-negative-values}

\begin{Shaded}
\begin{Highlighting}[]
\NormalTok{loan\_df }\OtherTok{\textless{}{-}}\NormalTok{ loan\_df[loan\_df}\SpecialCharTok{$}\NormalTok{residential\_assets\_value }\SpecialCharTok{\textgreater{}=} \DecValTok{0}\NormalTok{, ]}
\FunctionTok{str}\NormalTok{(loan\_df)}
\end{Highlighting}
\end{Shaded}

\begin{verbatim}
## 'data.frame':    4241 obs. of  12 variables:
##  $ no_of_dependents        : Factor w/ 6 levels "0","1","2","3",..: 3 1 4 4 6 1 6 3 1 6 ...
##  $ education               : Factor w/ 2 levels " Graduate"," Not Graduate": 1 2 1 1 2 1 1 1 1 2 ...
##  $ self_employed           : Factor w/ 2 levels " No"," Yes": 1 2 1 1 2 2 1 2 2 1 ...
##  $ income_annum            : int  9600000 4100000 9100000 8200000 9800000 4800000 8700000 5700000 800000 1100000 ...
##  $ loan_amount             : int  29900000 12200000 29700000 30700000 24200000 13500000 33000000 15000000 2200000 4300000 ...
##  $ loan_term               : int  12 8 20 8 20 10 4 20 20 10 ...
##  $ cibil_score             : int  778 417 506 467 382 319 678 382 782 388 ...
##  $ residential_assets_value: int  2400000 2700000 7100000 18200000 12400000 6800000 22500000 13200000 1300000 3200000 ...
##  $ commercial_assets_value : int  17600000 2200000 4500000 3300000 8200000 8300000 14800000 5700000 800000 1400000 ...
##  $ luxury_assets_value     : int  22700000 8800000 33300000 23300000 29400000 13700000 29200000 11800000 2800000 3300000 ...
##  $ bank_asset_value        : int  8000000 3300000 12800000 7900000 5000000 5100000 4300000 6000000 600000 1600000 ...
##  $ loan_status             : Factor w/ 2 levels " Approved"," Rejected": 1 2 2 2 2 2 1 2 1 2 ...
\end{verbatim}

\begin{Shaded}
\begin{Highlighting}[]
\FunctionTok{summary}\NormalTok{(loan\_df)}
\end{Highlighting}
\end{Shaded}

\begin{verbatim}
##  no_of_dependents         education    self_employed  income_annum    
##  0:706             Graduate    :2127    No :2106     Min.   : 200000  
##  1:696             Not Graduate:2114    Yes:2135     1st Qu.:2700000  
##  2:701                                               Median :5100000  
##  3:725                                               Mean   :5074251  
##  4:744                                               3rd Qu.:7500000  
##  5:669                                               Max.   :9900000  
##   loan_amount         loan_term     cibil_score  residential_assets_value
##  Min.   :  300000   Min.   : 2.0   Min.   :300   Min.   :       0        
##  1st Qu.: 7700000   1st Qu.: 6.0   1st Qu.:453   1st Qu.: 2200000        
##  Median :14600000   Median :10.0   Median :600   Median : 5700000        
##  Mean   :15178401   Mean   :10.9   Mean   :600   Mean   : 7522613        
##  3rd Qu.:21500000   3rd Qu.:16.0   3rd Qu.:747   3rd Qu.:11400000        
##  Max.   :39500000   Max.   :20.0   Max.   :900   Max.   :29100000        
##  commercial_assets_value luxury_assets_value bank_asset_value  
##  Min.   :       0        Min.   :  300000    Min.   :       0  
##  1st Qu.: 1300000        1st Qu.: 7500000    1st Qu.: 2400000  
##  Median : 3700000        Median :14600000    Median : 4600000  
##  Mean   : 4985121        Mean   :15171210    Mean   : 4991488  
##  3rd Qu.: 7700000        3rd Qu.:21700000    3rd Qu.: 7100000  
##  Max.   :19400000        Max.   :39200000    Max.   :14700000  
##     loan_status  
##   Approved:2640  
##   Rejected:1601  
##                  
##                  
##                  
## 
\end{verbatim}

After filtering out the rows with negative values, the dataset now has
\texttt{4,241\ observations}. The distributions across all variables
look almost identical to before, showing that this cleaning step didn't
affect the data overall.

\subsection{Exploratory Data Analysis}\label{exploratory-data-analysis}

\subsubsection{Does education level affect loan
approval?}\label{does-education-level-affect-loan-approval}

\subsubsection{Do people with higher asset values tend to get approved
more
often?}\label{do-people-with-higher-asset-values-tend-to-get-approved-more-often}

\subsubsection{Is there a significant difference in average income
between approved and rejected
applicants?}\label{is-there-a-significant-difference-in-average-income-between-approved-and-rejected-applicants}

\begin{Shaded}
\begin{Highlighting}[]
\CommentTok{\# Group by loan approval status and extract their annual income data}
\NormalTok{approved\_income }\OtherTok{\textless{}{-}}\NormalTok{ loan\_df}\SpecialCharTok{$}\NormalTok{income\_annum[loan\_df}\SpecialCharTok{$}\NormalTok{loan\_status }\SpecialCharTok{==} \StringTok{" Approved"}\NormalTok{]}
\NormalTok{rejected\_income }\OtherTok{\textless{}{-}}\NormalTok{ loan\_df}\SpecialCharTok{$}\NormalTok{income\_annum[loan\_df}\SpecialCharTok{$}\NormalTok{loan\_status }\SpecialCharTok{==} \StringTok{" Rejected"}\NormalTok{]}
\CommentTok{\# Calculate each group\textquotesingle{}s mean annual income}
\NormalTok{mean\_approved }\OtherTok{\textless{}{-}} \FunctionTok{mean}\NormalTok{(approved\_income, }\AttributeTok{na.rm =} \ConstantTok{TRUE}\NormalTok{)}
\NormalTok{mean\_rejected }\OtherTok{\textless{}{-}} \FunctionTok{mean}\NormalTok{(rejected\_income, }\AttributeTok{na.rm =} \ConstantTok{TRUE}\NormalTok{)}
\FunctionTok{cat}\NormalTok{(}\StringTok{"Mean annual income of Approved group:"}\NormalTok{, mean\_approved, }\StringTok{"}\SpecialCharTok{\textbackslash{}n}\StringTok{"}\NormalTok{)}
\end{Highlighting}
\end{Shaded}

\begin{verbatim}
## Mean annual income of Approved group: 5034886
\end{verbatim}

\begin{Shaded}
\begin{Highlighting}[]
\FunctionTok{cat}\NormalTok{(}\StringTok{"Mean annual income of Rejected group:"}\NormalTok{, mean\_rejected, }\StringTok{"}\SpecialCharTok{\textbackslash{}n}\StringTok{"}\NormalTok{)}
\end{Highlighting}
\end{Shaded}

\begin{verbatim}
## Mean annual income of Rejected group: 5139163
\end{verbatim}

\begin{Shaded}
\begin{Highlighting}[]
\CommentTok{\# Conduct hypothesis testing}
\NormalTok{t\_test\_result }\OtherTok{\textless{}{-}} \FunctionTok{t.test}\NormalTok{(approved\_income, rejected\_income)}
\FunctionTok{print}\NormalTok{(}\StringTok{"t{-}test result:"}\NormalTok{)}
\end{Highlighting}
\end{Shaded}

\begin{verbatim}
## [1] "t-test result:"
\end{verbatim}

\begin{Shaded}
\begin{Highlighting}[]
\FunctionTok{print}\NormalTok{(t\_test\_result)}
\end{Highlighting}
\end{Shaded}

\begin{verbatim}
## 
##  Welch Two Sample t-test
## 
## data:  approved_income and rejected_income
## t = -1, df = 3435, p-value = 0.2
## alternative hypothesis: true difference in means is not equal to 0
## 95 percent confidence interval:
##  -277421   68867
## sample estimates:
## mean of x mean of y 
##   5034886   5139163
\end{verbatim}

\begin{Shaded}
\begin{Highlighting}[]
\CommentTok{\# Draw the histogram}
\FunctionTok{hist}\NormalTok{(approved\_income, }
     \AttributeTok{col =} \StringTok{"lightgreen"}\NormalTok{,}
     \AttributeTok{main =} \StringTok{"The distribution of annual income"}\NormalTok{, }
     \AttributeTok{xlab =} \StringTok{"Annual income"}\NormalTok{, }
     \AttributeTok{ylab =} \StringTok{"Frequency"}\NormalTok{,}
     \AttributeTok{breaks =} \DecValTok{30}\NormalTok{)}
\FunctionTok{hist}\NormalTok{(rejected\_income, }
     \AttributeTok{col =} \StringTok{"red"}\NormalTok{, }
     \AttributeTok{breaks =} \DecValTok{30}\NormalTok{,}
     \AttributeTok{add =} \ConstantTok{TRUE}\NormalTok{)}
\FunctionTok{legend}\NormalTok{(}\StringTok{"topright"}\NormalTok{, }
       \AttributeTok{legend =} \FunctionTok{c}\NormalTok{(}\StringTok{"Approved"}\NormalTok{, }\StringTok{"Rejected"}\NormalTok{), }
       \AttributeTok{fill =} \FunctionTok{c}\NormalTok{(}\StringTok{"lightgreen"}\NormalTok{, }\StringTok{"red"}\NormalTok{)) }
\end{Highlighting}
\end{Shaded}

\pandocbounded{\includegraphics[keepaspectratio]{main_files/figure-latex/Q3-1.pdf}}
The average annual income of the approved group is \texttt{5,025,904},
while that of the rejected group is \texttt{5,113,825}, resulting in a
mean difference of approximately \texttt{87,921}. To determine whether
this difference is statistically significant, a Welch Two-Sample t-test
was performed.

\begin{itemize}
\item
  \textbf{Null Hypothesis (H₀):} There is no significant difference in
  the average annual income between approved and rejected applicants.
\item
  \textbf{Alternative Hypothesis (H₁):} There is a significant
  difference in the average annual income between approved and rejected
  applicants.
\end{itemize}

The test results showed a \textbf{t-value} of \texttt{-1} and a
\textbf{p-value} of \texttt{0.3}, which is much greater than the
significance level \texttt{(α\ =\ 0.05)}. This indicates that the
difference between the two means is \textbf{not statistically
significant}. The \texttt{95\%} confidence interval for the mean
difference is \texttt{{[}-260,821,\ 84,978{]}}, which includes zero;
further supporting the conclusion that the average incomes of the two
groups may be equal.

A histogram comparing the income distributions of both groups also shows
similar shapes, reinforcing that there is no clear distinction between
them.

In conclusion, income alone does not appear to be a strong predictor of
loan approval in this dataset.

\subsubsection{Is there a correlation between applicant's annual income
and loan amount
requested?}\label{is-there-a-correlation-between-applicants-annual-income-and-loan-amount-requested}

\begin{Shaded}
\begin{Highlighting}[]
\NormalTok{annual\_income }\OtherTok{\textless{}{-}}\NormalTok{ loan\_df}\SpecialCharTok{$}\NormalTok{income\_annum}
\NormalTok{loan\_amount }\OtherTok{\textless{}{-}}\NormalTok{ loan\_df}\SpecialCharTok{$}\NormalTok{loan\_amount}

\CommentTok{\# Check whether normal distribution}
\NormalTok{shapiro\_income }\OtherTok{\textless{}{-}} \FunctionTok{shapiro.test}\NormalTok{(annual\_income)}
\NormalTok{shapiro\_loan }\OtherTok{\textless{}{-}} \FunctionTok{shapiro.test}\NormalTok{(loan\_amount)}
\FunctionTok{cat}\NormalTok{(}\StringTok{"p{-}value of annual income:"}\NormalTok{, shapiro\_income}\SpecialCharTok{$}\NormalTok{p.value, }\StringTok{"}\SpecialCharTok{\textbackslash{}n}\StringTok{"}\NormalTok{)}
\end{Highlighting}
\end{Shaded}

\begin{verbatim}
## p-value of annual income: 7.7e-34
\end{verbatim}

\begin{Shaded}
\begin{Highlighting}[]
\FunctionTok{cat}\NormalTok{(}\StringTok{"p{-}value of loan amount:"}\NormalTok{, shapiro\_loan}\SpecialCharTok{$}\NormalTok{p.value, }\StringTok{"}\SpecialCharTok{\textbackslash{}n}\StringTok{"}\NormalTok{)}
\end{Highlighting}
\end{Shaded}

\begin{verbatim}
## p-value of loan amount: 6.22e-28
\end{verbatim}

\begin{Shaded}
\begin{Highlighting}[]
\CommentTok{\# Choose the correlation coefficient}
\ControlFlowTok{if}\NormalTok{ (shapiro\_income}\SpecialCharTok{$}\NormalTok{p.value }\SpecialCharTok{\textgreater{}} \FloatTok{0.05} \SpecialCharTok{\&}\NormalTok{ shapiro\_loan}\SpecialCharTok{$}\NormalTok{p.value }\SpecialCharTok{\textgreater{}} \FloatTok{0.05}\NormalTok{) \{}
\NormalTok{  cor\_result }\OtherTok{\textless{}{-}} \FunctionTok{cor.test}\NormalTok{(annual\_income, loan\_amount, }\AttributeTok{method =} \StringTok{"pearson"}\NormalTok{)}
  \FunctionTok{cat}\NormalTok{(}\StringTok{"Pearson result:}\SpecialCharTok{\textbackslash{}n}\StringTok{"}\NormalTok{)}
\NormalTok{\} }\ControlFlowTok{else}\NormalTok{ \{}
\NormalTok{  cor\_result }\OtherTok{\textless{}{-}} \FunctionTok{cor.test}\NormalTok{(annual\_income, loan\_amount, }\AttributeTok{method =} \StringTok{"spearman"}\NormalTok{)}
  \FunctionTok{cat}\NormalTok{(}\StringTok{"Spearman reslut:}\SpecialCharTok{\textbackslash{}n}\StringTok{"}\NormalTok{)}
\NormalTok{\}}
\end{Highlighting}
\end{Shaded}

\begin{verbatim}
## Spearman reslut:
\end{verbatim}

\begin{Shaded}
\begin{Highlighting}[]
\FunctionTok{print}\NormalTok{(cor\_result)}
\end{Highlighting}
\end{Shaded}

\begin{verbatim}
## 
##  Spearman's rank correlation rho
## 
## data:  annual_income and loan_amount
## S = 8e+08, p-value <2e-16
## alternative hypothesis: true rho is not equal to 0
## sample estimates:
##   rho 
## 0.941
\end{verbatim}

\begin{Shaded}
\begin{Highlighting}[]
\CommentTok{\# Draw the plot}
\FunctionTok{plot}\NormalTok{(}
  \AttributeTok{x =}\NormalTok{ annual\_income,}
  \AttributeTok{y =}\NormalTok{ loan\_amount,}
  \AttributeTok{main =} \StringTok{"Annual Income vs. Loan Amount Requested"}\NormalTok{,}
  \AttributeTok{xlab =} \StringTok{"Annual Income"}\NormalTok{,}
  \AttributeTok{ylab =} \StringTok{"Loan Amount Requested"}\NormalTok{,}
  \AttributeTok{pch =} \DecValTok{16}\NormalTok{,}
  \AttributeTok{col =} \StringTok{"lightblue"}\NormalTok{,}
  \AttributeTok{cex =} \FloatTok{0.8}
\NormalTok{)}
\CommentTok{\# Add the regression line}
\FunctionTok{abline}\NormalTok{(}\FunctionTok{lm}\NormalTok{(loan\_amount }\SpecialCharTok{\textasciitilde{}}\NormalTok{ annual\_income), }\AttributeTok{col =} \StringTok{"red"}\NormalTok{, }\AttributeTok{lwd =} \DecValTok{2}\NormalTok{)}
\FunctionTok{legend}\NormalTok{(}\StringTok{"topleft"}\NormalTok{, }
       \AttributeTok{legend =} \StringTok{"regression line"}\NormalTok{, }
       \AttributeTok{col =} \StringTok{"red"}\NormalTok{, }
       \AttributeTok{lty =} \DecValTok{1}\NormalTok{, }
       \AttributeTok{lwd =} \DecValTok{2}\NormalTok{)}
\end{Highlighting}
\end{Shaded}

\pandocbounded{\includegraphics[keepaspectratio]{main_files/figure-latex/Q4-1.pdf}}
I checked whether the data of annual income and loan amount is follow
the normal distribution and choose the Spearman correlation coefficient
according to the result. The rho is 0.941 which is close to 1,
indicating that there is a very strong positive monotonic correlation
between the annual income and loan amount exist and the p-value is far
less than 0.05 also showed this correlation is very significant in
statistic. Then according to the plot chart, it's very clear that the
loan amount shows an obvious upward trend from the bottom left to the
top right as annual income increases, and the red regression line
further strengthens the visual evidence of this linear association,
indicating a strong positive correlation tendency between the two. In
short, this analysis clearly draws the conclusion that ``annual income
is a key factor affecting the loan application amount, and there is a
strong positive correlation between the two.''

\subsubsection{Is there a significant difference in average CIBIL scores
between approved and rejected
applicants?}\label{is-there-a-significant-difference-in-average-cibil-scores-between-approved-and-rejected-applicants}

\begin{Shaded}
\begin{Highlighting}[]
\FunctionTok{library}\NormalTok{(ggplot2)}
\CommentTok{\#CIBIL score frequency distribution}
\FunctionTok{hist}\NormalTok{(loan\_df}\SpecialCharTok{$}\NormalTok{cibil\_score, }\AttributeTok{main =} \StringTok{"Frequency Distribution of CIBIL Scores"}\NormalTok{, }\AttributeTok{xlab =} \StringTok{"CIBIL Score"}\NormalTok{, }\AttributeTok{ylab =} \StringTok{"Frequency"}\NormalTok{, }\AttributeTok{breaks =} \DecValTok{15}\NormalTok{, }\AttributeTok{col =} \StringTok{"lightblue"}\NormalTok{)}
\end{Highlighting}
\end{Shaded}

\pandocbounded{\includegraphics[keepaspectratio]{main_files/figure-latex/Q5-1.pdf}}

\begin{Shaded}
\begin{Highlighting}[]
\CommentTok{\#CIBIL score frequency distribution}
\NormalTok{loan\_df}\SpecialCharTok{$}\NormalTok{loan\_status }\OtherTok{\textless{}{-}} \FunctionTok{factor}\NormalTok{(}\FunctionTok{trimws}\NormalTok{(loan\_df}\SpecialCharTok{$}\NormalTok{loan\_status))}
\FunctionTok{ggplot}\NormalTok{(loan\_df, }\FunctionTok{aes}\NormalTok{(}\AttributeTok{x =}\NormalTok{ cibil\_score, }\AttributeTok{fill =}\NormalTok{ loan\_status)) }\SpecialCharTok{+}
  \FunctionTok{geom\_histogram}\NormalTok{(}\AttributeTok{binwidth =} \DecValTok{50}\NormalTok{, }\AttributeTok{color =} \StringTok{"white"}\NormalTok{, }\AttributeTok{alpha =} \FloatTok{0.8}\NormalTok{) }\SpecialCharTok{+}
  \FunctionTok{facet\_wrap}\NormalTok{(}\SpecialCharTok{\textasciitilde{}}\NormalTok{ loan\_status, }\AttributeTok{scales =} \StringTok{"free\_y"}\NormalTok{) }\SpecialCharTok{+}
  \FunctionTok{labs}\NormalTok{(}\AttributeTok{title =} \StringTok{"CIBIL Score Distribution by Loan Status"}\NormalTok{, }\AttributeTok{x =} \StringTok{"CIBIL Score"}\NormalTok{, }\AttributeTok{y =} \StringTok{"Frequency"}\NormalTok{, }\AttributeTok{fill =} \StringTok{"Loan Status"}\NormalTok{) }\SpecialCharTok{+} \FunctionTok{scale\_fill\_manual}\NormalTok{(}\AttributeTok{values =} \FunctionTok{c}\NormalTok{(}\StringTok{"Approved"} \OtherTok{=} \StringTok{"darkgreen"}\NormalTok{, }\StringTok{"Rejected"} \OtherTok{=} \StringTok{"darkred"}\NormalTok{))}
\end{Highlighting}
\end{Shaded}

\pandocbounded{\includegraphics[keepaspectratio]{main_files/figure-latex/Q5-2.pdf}}

\begin{Shaded}
\begin{Highlighting}[]
\CommentTok{\#CIBIL score frequency box chart}
\FunctionTok{ggplot}\NormalTok{(loan\_df, }\FunctionTok{aes}\NormalTok{(}\AttributeTok{x =}\NormalTok{ loan\_status, }\AttributeTok{y =}\NormalTok{ cibil\_score, }\AttributeTok{fill =}\NormalTok{ loan\_status)) }\SpecialCharTok{+}
  \FunctionTok{geom\_boxplot}\NormalTok{() }\SpecialCharTok{+}
  \FunctionTok{labs}\NormalTok{(}\AttributeTok{title =} \StringTok{"CIBIL Score Distribution by Loan Status"}\NormalTok{, }\AttributeTok{x =} \StringTok{"Loan Status"}\NormalTok{, }\AttributeTok{y =} \StringTok{"CIBIL Score"}\NormalTok{) }\SpecialCharTok{+}
  \FunctionTok{scale\_fill\_manual}\NormalTok{(}\AttributeTok{values =} \FunctionTok{c}\NormalTok{(}\StringTok{"Approved"} \OtherTok{=} \StringTok{"darkgreen"}\NormalTok{, }\StringTok{"Rejected"} \OtherTok{=} \StringTok{"darkred"}\NormalTok{))}
\end{Highlighting}
\end{Shaded}

\pandocbounded{\includegraphics[keepaspectratio]{main_files/figure-latex/Q5-3.pdf}}

\begin{Shaded}
\begin{Highlighting}[]
\CommentTok{\#t{-}test}
\FunctionTok{t.test}\NormalTok{(cibil\_score }\SpecialCharTok{\textasciitilde{}}\NormalTok{ loan\_status, }\AttributeTok{data =}\NormalTok{ loan\_df)}
\end{Highlighting}
\end{Shaded}

\begin{verbatim}
## 
##  Welch Two Sample t-test
## 
## data:  cibil_score by loan_status
## t = 88, df = 4238, p-value <2e-16
## alternative hypothesis: true difference in means between group Approved and group Rejected is not equal to 0
## 95 percent confidence interval:
##  268 280
## sample estimates:
## mean in group Approved mean in group Rejected 
##                    703                    429
\end{verbatim}

The overall CIBIL score frequency distribution shows no apparent
outliers. The individual distributions for approved and rejected loans
show a few outliers, but it is not necessary to remove them. Even though
the outliers are dragging the two means closer to each other, the means
are still significantly different even when including them in the
t-test.

There is a significant difference in CIBIL scores between approved and
rejected applicants because the p-value, 2e-16, is much lower than the
standard alpha threshold of 0.05. This allows us to reject the null
hypothesis that the two means of approved and rejected applicants are
equal.

\subsubsection{For rejected loans, is there a significant difference in
the mean CIBIL score between applicants with shorter loan term loans
(\textless=10 years) versus longer term loans(\textgreater10
years)?}\label{for-rejected-loans-is-there-a-significant-difference-in-the-mean-cibil-score-between-applicants-with-shorter-loan-term-loans-10-years-versus-longer-term-loans10-years}

\begin{Shaded}
\begin{Highlighting}[]
\CommentTok{\#created subsets for shorter and longer loans then ran t{-}test}
\NormalTok{rejected\_shorter }\OtherTok{\textless{}{-}} \FunctionTok{subset}\NormalTok{(loan\_df, }\FunctionTok{trimws}\NormalTok{(loan\_status) }\SpecialCharTok{==} \StringTok{"Rejected"} \SpecialCharTok{\&}\NormalTok{ loan\_term }\SpecialCharTok{\textless{}=} \DecValTok{10}\NormalTok{)}
\NormalTok{rejected\_longer }\OtherTok{\textless{}{-}} \FunctionTok{subset}\NormalTok{(loan\_df, }\FunctionTok{trimws}\NormalTok{(loan\_status) }\SpecialCharTok{==} \StringTok{"Rejected"} \SpecialCharTok{\&}\NormalTok{ loan\_term }\SpecialCharTok{\textgreater{}} \DecValTok{10}\NormalTok{)}

\CommentTok{\#Overall loan term frequency distribution}
\FunctionTok{ggplot}\NormalTok{(loan\_df, }\FunctionTok{aes}\NormalTok{(}\AttributeTok{x =} \FunctionTok{factor}\NormalTok{(loan\_term))) }\SpecialCharTok{+} \FunctionTok{geom\_bar}\NormalTok{(}\AttributeTok{fill =} \StringTok{"lightblue"}\NormalTok{) }\SpecialCharTok{+} \FunctionTok{labs}\NormalTok{(}\AttributeTok{title =} \StringTok{"Frequency Distribution of Loan Terms"}\NormalTok{, }\AttributeTok{x =} \StringTok{"Loan Term (Years)"}\NormalTok{, }\AttributeTok{y =} \StringTok{"Frequency"}\NormalTok{)}
\end{Highlighting}
\end{Shaded}

\pandocbounded{\includegraphics[keepaspectratio]{main_files/figure-latex/Q6-1.pdf}}

\begin{Shaded}
\begin{Highlighting}[]
\CommentTok{\#boxplot}
\FunctionTok{boxplot}\NormalTok{(rejected\_shorter}\SpecialCharTok{$}\NormalTok{cibil\_score, rejected\_longer}\SpecialCharTok{$}\NormalTok{cibil\_score, }\AttributeTok{names =} \FunctionTok{c}\NormalTok{(}\StringTok{"Shorter (\textless{}= 10 yrs)"}\NormalTok{, }\StringTok{"Longer (\textgreater{} 10 yrs)"}\NormalTok{), }\AttributeTok{main =} \StringTok{"CIBIL Score Distribution for Rejected Loans"}\NormalTok{, }\AttributeTok{ylab =} \StringTok{"CIBIL Score"}\NormalTok{, }\AttributeTok{xlab =} \StringTok{"Loan Term Group"}\NormalTok{)}
\end{Highlighting}
\end{Shaded}

\pandocbounded{\includegraphics[keepaspectratio]{main_files/figure-latex/Q6-2.pdf}}

\begin{Shaded}
\begin{Highlighting}[]
\CommentTok{\#removing outliers}
\NormalTok{removeOutliers }\OtherTok{\textless{}{-}} \ControlFlowTok{function}\NormalTok{(x) \{Q1 }\OtherTok{\textless{}{-}} \FunctionTok{quantile}\NormalTok{(x, }\FloatTok{0.25}\NormalTok{)}
\NormalTok{Q3 }\OtherTok{\textless{}{-}} \FunctionTok{quantile}\NormalTok{(x, }\FloatTok{0.75}\NormalTok{);}
\NormalTok{IQR\_val }\OtherTok{\textless{}{-}}\NormalTok{ Q3 }\SpecialCharTok{{-}}\NormalTok{ Q1;}
\NormalTok{lower\_bound }\OtherTok{\textless{}{-}}\NormalTok{ Q1 }\SpecialCharTok{{-}}\NormalTok{ (}\FloatTok{1.5} \SpecialCharTok{*}\NormalTok{ IQR\_val);}
\NormalTok{upper\_bound }\OtherTok{\textless{}{-}}\NormalTok{ Q3 }\SpecialCharTok{+}\NormalTok{ (}\FloatTok{1.5} \SpecialCharTok{*}\NormalTok{ IQR\_val);}
  
\FunctionTok{return}\NormalTok{(x[x }\SpecialCharTok{\textgreater{}=}\NormalTok{ lower\_bound }\SpecialCharTok{\&}\NormalTok{ x }\SpecialCharTok{\textless{}=}\NormalTok{ upper\_bound]);\}}

\NormalTok{cleaned\_shorter\_cibil }\OtherTok{\textless{}{-}} \FunctionTok{removeOutliers}\NormalTok{(rejected\_shorter}\SpecialCharTok{$}\NormalTok{cibil\_score)}
\NormalTok{cleaned\_longer\_cibil }\OtherTok{\textless{}{-}} \FunctionTok{removeOutliers}\NormalTok{(rejected\_longer}\SpecialCharTok{$}\NormalTok{cibil\_score)}

\FunctionTok{t.test}\NormalTok{(rejected\_shorter}\SpecialCharTok{$}\NormalTok{cibil\_score, rejected\_longer}\SpecialCharTok{$}\NormalTok{cibil\_score)}
\end{Highlighting}
\end{Shaded}

\begin{verbatim}
## 
##  Welch Two Sample t-test
## 
## data:  rejected_shorter$cibil_score and rejected_longer$cibil_score
## t = -0.03, df = 1564, p-value = 1
## alternative hypothesis: true difference in means is not equal to 0
## 95 percent confidence interval:
##  -7.72  7.51
## sample estimates:
## mean of x mean of y 
##       429       429
\end{verbatim}

\begin{Shaded}
\begin{Highlighting}[]
\FunctionTok{t.test}\NormalTok{(cleaned\_shorter\_cibil, cleaned\_longer\_cibil)}
\end{Highlighting}
\end{Shaded}

\begin{verbatim}
## 
##  Welch Two Sample t-test
## 
## data:  cleaned_shorter_cibil and cleaned_longer_cibil
## t = -0.05, df = 1563, p-value = 1
## alternative hypothesis: true difference in means is not equal to 0
## 95 percent confidence interval:
##  -7.47  7.09
## sample estimates:
## mean of x mean of y 
##       427       428
\end{verbatim}

The overall loan terms frequency distribution shows no apparent outliers
however there are a few in the shorter and longer loan term groups.
Having this said, even when removing these outliers there no significant
differences found between the two groups and their respective CIBIL
scores.

There is not a significant difference in CIBIL scores between shorter
(\textless= 10 years) and longer (\textgreater10 years) term loans
within the rejected group of applicants because the p-value, 1, is much
higher than the standard alpha threshold of 0.05. This allows us to
accept the null hypothesis that the two means of shorter and longer term
loans of rejected applicants are equal.

\subsubsection{Is there a correlation between the CIBIL Score and the
Loan Term among those whose loans are
approved?}\label{is-there-a-correlation-between-the-cibil-score-and-the-loan-term-among-those-whose-loans-are-approved}

\begin{Shaded}
\begin{Highlighting}[]
\CommentTok{\#correlation coefficient}
\NormalTok{corr\_r }\OtherTok{\textless{}{-}} \FunctionTok{cor}\NormalTok{(loan\_df[}\FunctionTok{trimws}\NormalTok{(loan\_df}\SpecialCharTok{$}\NormalTok{loan\_status) }\SpecialCharTok{==} \StringTok{"Approved"}\NormalTok{, }\StringTok{"cibil\_score"}\NormalTok{], loan\_df[}\FunctionTok{trimws}\NormalTok{(loan\_df}\SpecialCharTok{$}\NormalTok{loan\_status) }\SpecialCharTok{==} \StringTok{"Approved"}\NormalTok{, }\StringTok{"loan\_term"}\NormalTok{], }\AttributeTok{method =} \StringTok{"pearson"}\NormalTok{)}

\CommentTok{\#correlation test}
\FunctionTok{cor.test}\NormalTok{(loan\_df[}\FunctionTok{trimws}\NormalTok{(loan\_df}\SpecialCharTok{$}\NormalTok{loan\_status) }\SpecialCharTok{==} \StringTok{"Approved"}\NormalTok{, }\StringTok{"cibil\_score"}\NormalTok{], loan\_df[}\FunctionTok{trimws}\NormalTok{(loan\_df}\SpecialCharTok{$}\NormalTok{loan\_status) }\SpecialCharTok{==} \StringTok{"Approved"}\NormalTok{, }\StringTok{"loan\_term"}\NormalTok{], }\AttributeTok{method =} \StringTok{"pearson"}\NormalTok{)}
\end{Highlighting}
\end{Shaded}

\begin{verbatim}
## 
##  Pearson's product-moment correlation
## 
## data:  loan_df[trimws(loan_df$loan_status) == "Approved", "cibil_score"] and loan_df[trimws(loan_df$loan_status) == "Approved", "loan_term"]
## t = 11, df = 2638, p-value <2e-16
## alternative hypothesis: true correlation is not equal to 0
## 95 percent confidence interval:
##  0.173 0.246
## sample estimates:
##  cor 
## 0.21
\end{verbatim}

\begin{Shaded}
\begin{Highlighting}[]
\CommentTok{\#scatter plot}
\FunctionTok{ggplot}\NormalTok{(}\FunctionTok{subset}\NormalTok{(loan\_df, }\FunctionTok{trimws}\NormalTok{(loan\_status) }\SpecialCharTok{==} \StringTok{"Approved"}\NormalTok{), }\FunctionTok{aes}\NormalTok{(}\AttributeTok{x =}\NormalTok{ loan\_term, }\AttributeTok{y =}\NormalTok{ cibil\_score)) }\SpecialCharTok{+} \FunctionTok{geom\_point}\NormalTok{(}\AttributeTok{alpha =} \FloatTok{0.4}\NormalTok{, }\AttributeTok{color =} \StringTok{"blue"}\NormalTok{) }\SpecialCharTok{+} \FunctionTok{geom\_smooth}\NormalTok{(}\AttributeTok{method =} \StringTok{"lm"}\NormalTok{, }\AttributeTok{se =} \ConstantTok{FALSE}\NormalTok{, }\AttributeTok{color =} \StringTok{"red"}\NormalTok{) }\SpecialCharTok{+} \FunctionTok{labs}\NormalTok{(}\AttributeTok{title =} \StringTok{"CIBIL Score vs. Loan Term for Approved Loans"}\NormalTok{, }\AttributeTok{x =} \StringTok{"Loan Term (Years)"}\NormalTok{, }\AttributeTok{y =} \StringTok{"CIBIL Score"}\NormalTok{)}
\end{Highlighting}
\end{Shaded}

\pandocbounded{\includegraphics[keepaspectratio]{main_files/figure-latex/Q7-1.pdf}}

The correlation coefficient of 0.21 shows a weak but positive
relationship where approved candidates with higher CIBIL scores have a
slightly higher tendency to approved for longer loan terms. And the
p-value is very small, 2e-16, meaning that the correlation coefficient
is unlikely due to chance.

\subsubsection{Does self-employment status affect loan
approval?}\label{does-self-employment-status-affect-loan-approval}

\begin{Shaded}
\begin{Highlighting}[]
\FunctionTok{ggplot}\NormalTok{(loan\_df, }\FunctionTok{aes}\NormalTok{(}\AttributeTok{x =}\NormalTok{ self\_employed, }\AttributeTok{fill =}\NormalTok{ loan\_status)) }\SpecialCharTok{+}
  \FunctionTok{geom\_bar}\NormalTok{(}\AttributeTok{position =} \StringTok{"dodge"}\NormalTok{) }\SpecialCharTok{+}
  \FunctionTok{labs}\NormalTok{(}
    \AttributeTok{title =} \StringTok{"Loan Approval by Number of Dependents"}\NormalTok{,}
    \AttributeTok{x =} \StringTok{"Number of Dependents"}\NormalTok{,}
    \AttributeTok{y =} \StringTok{"Number of Applicants"}
\NormalTok{  ) }\SpecialCharTok{+}
  \FunctionTok{scale\_fill\_manual}\NormalTok{(}\AttributeTok{values =} \FunctionTok{c}\NormalTok{(}\StringTok{"Approved"} \OtherTok{=} \StringTok{"\#F2764E"}\NormalTok{, }\AttributeTok{Rejected =} \StringTok{"\#FAF487"}\NormalTok{)) }\SpecialCharTok{+}
  \FunctionTok{theme\_minimal}\NormalTok{()}
\end{Highlighting}
\end{Shaded}

\pandocbounded{\includegraphics[keepaspectratio]{main_files/figure-latex/Q8-1.pdf}}
Approval rates are nearly \textbf{identical} for self-employed and non
self-employed applicants, suggesting self-employment likely doesn't
affect loan approval. A Chi-square test will confirm this statistically.

\begin{Shaded}
\begin{Highlighting}[]
\NormalTok{contable }\OtherTok{\textless{}{-}} \FunctionTok{table}\NormalTok{(loan\_df}\SpecialCharTok{$}\NormalTok{self\_employed, loan\_df}\SpecialCharTok{$}\NormalTok{loan\_status)}
\FunctionTok{chisq.test}\NormalTok{(contable)}
\end{Highlighting}
\end{Shaded}

\begin{verbatim}
## 
##  Pearson's Chi-squared test with Yates' continuity correction
## 
## data:  contable
## X-squared = 0.009, df = 1, p-value = 0.9
\end{verbatim}

The Chi-square test (\texttt{p\ =\ 1}) also shows no significant
association between self-employment status and loan approval. This
confirms that self-employment does not affect the likelihood of loan
approval in this dataset.

\subsubsection{Does the number of dependents affect approval of
loans?}\label{does-the-number-of-dependents-affect-approval-of-loans}

\begin{Shaded}
\begin{Highlighting}[]
\FunctionTok{ggplot}\NormalTok{(loan\_df, }\FunctionTok{aes}\NormalTok{(}\AttributeTok{x =}\NormalTok{ no\_of\_dependents, }\AttributeTok{fill =}\NormalTok{ loan\_status)) }\SpecialCharTok{+}
  \FunctionTok{geom\_bar}\NormalTok{(}\AttributeTok{position =} \StringTok{"dodge"}\NormalTok{) }\SpecialCharTok{+}
  \FunctionTok{labs}\NormalTok{(}
    \AttributeTok{title =} \StringTok{"Loan Approval by Number of Dependents"}\NormalTok{,}
    \AttributeTok{x =} \StringTok{"Number of Dependents"}\NormalTok{,}
    \AttributeTok{y =} \StringTok{"Number of Applicants"}
\NormalTok{  ) }\SpecialCharTok{+}
  \FunctionTok{scale\_fill\_manual}\NormalTok{(}\AttributeTok{values =} \FunctionTok{c}\NormalTok{(}\StringTok{"Approved"} \OtherTok{=} \StringTok{"\#0D0942"}\NormalTok{, }\StringTok{"Rejected"} \OtherTok{=} \StringTok{"\#C4C0F6"}\NormalTok{)) }\SpecialCharTok{+}
  \FunctionTok{theme\_minimal}\NormalTok{()}
\end{Highlighting}
\end{Shaded}

\pandocbounded{\includegraphics[keepaspectratio]{main_files/figure-latex/Q9-1.pdf}}
The bar chart shows that approval and rejection counts are fairly
similar across all dependent categories, with no noticeable trend
suggesting that the number of dependents strongly influences loan
approval outcomes. While minor variations exist, the approval rate
appears consistent across groups, indicating that loan approval is
likely independent of the number of dependents. A Chi-square test will
confirm this statistically.

\begin{Shaded}
\begin{Highlighting}[]
\NormalTok{contable }\OtherTok{\textless{}{-}} \FunctionTok{table}\NormalTok{(loan\_df}\SpecialCharTok{$}\NormalTok{no\_of\_dependents, loan\_df}\SpecialCharTok{$}\NormalTok{loan\_status)}
\FunctionTok{chisq.test}\NormalTok{(contable)}
\end{Highlighting}
\end{Shaded}

\begin{verbatim}
## 
##  Pearson's Chi-squared test
## 
## data:  contable
## X-squared = 2, df = 5, p-value = 0.8
\end{verbatim}

The Chi-square test (\texttt{p\ =\ 0.8}) also shows no significant
association between no\_of\_dependents and loan approval. This confirms
that no\_of\_dependents does not affect the likelihood of loan approval
in this dataset.

\subsubsection{Does requested loan term affect
approval?}\label{does-requested-loan-term-affect-approval}

\begin{Shaded}
\begin{Highlighting}[]
\FunctionTok{ggplot}\NormalTok{(loan\_df, }\FunctionTok{aes}\NormalTok{(}\AttributeTok{x =}\NormalTok{ loan\_status, }\AttributeTok{y =}\NormalTok{ loan\_term, }\AttributeTok{fill =}\NormalTok{ loan\_status)) }\SpecialCharTok{+}
  \FunctionTok{geom\_boxplot}\NormalTok{(}\AttributeTok{alpha =} \FloatTok{0.6}\NormalTok{) }\SpecialCharTok{+}
  \FunctionTok{labs}\NormalTok{(}
    \AttributeTok{title =} \StringTok{"Loan Term by Loan Approval Status"}\NormalTok{,}
    \AttributeTok{x =} \StringTok{"Loan Status"}\NormalTok{,}
    \AttributeTok{y =} \StringTok{"Loan Term (Years)"}
\NormalTok{  ) }\SpecialCharTok{+}
  \FunctionTok{scale\_fill\_manual}\NormalTok{(}\AttributeTok{values =} \FunctionTok{c}\NormalTok{(}\StringTok{"Approved"} \OtherTok{=} \StringTok{"\#98C25F"}\NormalTok{, }\StringTok{"Rejected"} \OtherTok{=} \StringTok{"\#EB3BA7"}\NormalTok{)) }\SpecialCharTok{+}
  \FunctionTok{theme\_minimal}\NormalTok{()}
\end{Highlighting}
\end{Shaded}

\pandocbounded{\includegraphics[keepaspectratio]{main_files/figure-latex/Q10-1.pdf}}
The boxplot shows that approved loans have a \textbf{wider range} of
terms, including very short loans, while rejected loans generally have
longer terms. The median loan term for approved loans is
\texttt{10\ years}, compared to \texttt{12\ years} for rejected loans,
indicating that rejected loans tend to have slightly longer terms

\begin{Shaded}
\begin{Highlighting}[]
\FunctionTok{t.test}\NormalTok{(loan\_term }\SpecialCharTok{\textasciitilde{}}\NormalTok{ loan\_status, }\AttributeTok{data =}\NormalTok{ loan\_df)}
\end{Highlighting}
\end{Shaded}

\begin{verbatim}
## 
##  Welch Two Sample t-test
## 
## data:  loan_term by loan_status
## t = -8, df = 3637, p-value = 2e-14
## alternative hypothesis: true difference in means between group Approved and group Rejected is not equal to 0
## 95 percent confidence interval:
##  -1.69 -1.01
## sample estimates:
## mean in group Approved mean in group Rejected 
##                   10.4                   11.7
\end{verbatim}

The two-sample t-test indicates a significant difference in means
(\texttt{p\ \textless{}\ 0.05}), with approved loans having a shorter
average term (\texttt{10.4\ years}) than rejected loans
(\texttt{11.7\ years}). The \texttt{95\%} confidence interval for the
mean difference is \texttt{-1.69\ to\ -1.01} (the interval doesn't have
0), confirming that approved loans tend to have shorter terms.

\end{document}
